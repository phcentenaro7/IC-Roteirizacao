\section{Introdução}
Este trabalho foi desenvolvido, de abril de 2022 a agosto de 2023, pelo discente Pedro Henrique Centenaro, estudante de Engenharia de Controle, Automação e Computação na Universidade Federal de Santa Catarina (UFSC), Campus Blumenau. O professor Luiz Rafael dos Santos atuou como orientador do projeto, e foi fornecido ao estudante acesso ao Laboratório de Matemática Aplicada e Computacional da UFSC, o LABMAC

A roteirização de veículos é um processo logístico fundamental para empresas. Entre suas aplicações, podem ser citadas a coleta de materiais recicláveis \cite{BAPTISTA:02}, coleta de lixo \cite{MARKOVIC:20}, distribuição de alimentos perecíveis \cite{AMORIM:14}, roteirização de atendimento médico \cite{ISSABAKHSH:18} e roteirização de ônibus \cite{GUO:19}. Os modelos matemáticos aplicados à resolução de tais problemas têm formulações variadas, cada qual dependendo da complexidade da análise realizada pelos autores.

O problema mais simples de roteirização é o do caixeiro viajante (PCV), em que um único veículo deve atender determinado número de pontos de modo a minimizar seus gastos com deslocamentos \cite{OR-Tools,SAIYED:12,SIQUEIRA:22}. Quando se considera uma frota, tem-se um problema de roteamento de veículos (PRV) \cite{OR-Tools-VRP}. Neste caso, é possível considerar uma quantidade muito maior de obstáculos à roteirização, como capacidades dos veículos, janelas de tempo de atendimento, múltiplos depósitos, frotas mistas e outros mais \cite{VIEIRA:13,TORO:16}.

Com o crescimento dos modelos dos problemas de roteirização, aumenta a complexidade computacional para resolvê-los. Não é incomum que se discutam soluções para PCVs relativamente pequenos que encontrem a solução ótima do problema; no entanto, para os PRVs, a prática exige que soluções quase-ótimas sejam consideradas \cite{LAPORTE:87}. Ainda assim, tanto o PCV quanto os PRVs são problemas NP-difíceis \cite{ZAMBITO:06}. Por exemplo, a resolução de um PCV de 85900 pontos necessitou do uso de 96 estações de trabalho e um poder computacional que equivale a 139 anos de funcionamento de uma única CPU \cite{BAZRAFSHAN:21}.

Tendo em vista a complexidade teórica e prática dos PRVs, o objetivo geral deste trabalho é explorar e explicar as principais características que contribuem para as dificuldades de implementação destes problemas. Já os objetivos específicos são:

\begin{itemize}
    \item Estudar as teorias dos problemas de roteamento de veículos capacitado e com janela de tempo, incluindo aplicações;
    \item Modelar PRVJT escrevendo as equações relacionadas e implementar o modelo na linguagem de modelagem JuMP;
    \item Fazer testes computacionais com diversos \emph{solvers}, em particular que utilizem técnicas exatas e heurísticas/meta-heurísticas;
    \item Reportar os resultados em formato de relatório técnico e pôster em eventos científicos.
\end{itemize}

Visando alcançar tais objetivos, os princípios matemáticos e computacionais necessários para a compreensão e implementação de modelos de roteamento foram concatenados neste relatório. A ideia é que o público-alvo seja capaz de ler o trabalho por completo, chegando ao fim com a capacidade de interpretar modelos complexos de PRVs e identificar as vantagens e desvantagens computacionais de cada tipo de formulação.

Para realizar a leitura deste documento, recomenda-se um conhecimento básico sobre demonstrações e notações matemáticas, além de álgebra linear. Na \cref{sec:urbanização e transporte público}, o transporte público urbano, seus desafios e tendências futuras são apresentados como motivadores para o estudo de PRVs. Na \cref{sec:matemática}, apresenta-se o básico de grafos, seguido de uma definição matemática formal do PCV. Pelo resto da seção, discutem-se objetos da álgebra linear necessários para compreender a ideia por trás do método simplex, muito utilizado para resolver problemas de otimização. Os mecanismos por trás deste método são discutidos em detalhes no \cref{sec:metodo_simplex}, incluindo pseudocódigo e implementação \cite{Centenaro:23} (ver o pacote \href{https://github.com/phcentenaro7/Caique.jl}{\texttt{Caique.jl}}).

A \cref{sec:considerações computacionais} lida com os aspectos computacionais de implementação e resolução de modelos. São comparados \emph{solvers}\footnote{Como é detalhado em seções posteriores, os \emph{solvers} são ferramentas computacionais robustas que tentam resolver problemas de otimização por meio de várias técnicas exatas e heurísticas.} e formulações do PCV, levando em conta a teoria desenvolvida até o momento. Detalhes de implementação podem ser encontrados no \cref{sec:PCV implementações}. Na \cref{sec:PRVJT}, são apresentados e descritos alguns modelos de PRVJTs. Similaridades e diferenças notáveis entre os modelos são comparadas, realçando o impacto que as considerações específicas de cada um têm sobre seu formato matemático e suas características computacionais. Finalmente, na \cref{sec:conclusão}, os objetivos deste trabalho são revisitados de modo a avaliar se e como foram cumpridos.