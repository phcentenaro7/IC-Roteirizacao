\section{Conclusão}\label{sec:conclusão}
O principal objetivo deste trabalho era o estudo teórico dos problemas de roteamento de veículos capacitado e com janelas de tempo, incluindo aplicações. Para este propósito, diversos empenhos foram empregados, a começar pelo estudo de diferentes disciplinas do curso de matemática da UFSC Blumenau. O estudante já havia tido um primeiro contato com a linguagem Julia e resolução computacional de problemas matemáticos na disciplina de Métodos Numéricos (\href{https://dmatbnu.paginas.ufsc.br/files/2021/05/MAT1831-08751-M\%C3\%A9todos_Num\%C3\%A9ricos_Plano_de_ensino_2021_1_MTM.pdf}{MAT1831}), antes de iniciar a pesquisa. Em semestres posteriores, estudou Otimização Contínua (\href{https://dmatbnu.paginas.ufsc.br/files/2022/03/1647624179733_MAT1032-Plano_de_Ensino_2022_1_Otimizacao_Continua.pdf}{MAT1032}) e Fundamentos da Matemática (\href{https://dmatbnu.paginas.ufsc.br/files/2022/07/MAT1121-Plano_de_Ensino_2022_2_Fundamentos_Matema\%CC\%81tica-MAT.pdf}{MAT1121}). Todas as disciplinas foram ministradas pelo orientador de pesquisa.

Concomitantemente, o estudante se dedicou à leitura de livros de álgebra linear e otimização linear, expondo-se ao aprendizado de ferramentas e técnicas matemáticas que estão além daquilo que é abordado em seu curso de engenharia. Para solidificar os conhecimentos adquiridos, implementou o método simplex em ambiente computacional \cite{Centenaro:23} e aprendeu a realizar a modelagem de problemas de otimização utilizando Julia \cite{JULIA} e o pacote JuMP \cite{JuMP}. Ao mesmo tempo, documentou os resultados das leituras, implementações e testes computacionais neste relatório. Durante a condução da pesquisa, por diversas vezes este documento foi reescrito e reorganizado, visando concatenar da melhor maneira possível as ideias conducentes à compreensão de PRVs.

Pode-se afirmar que este objetivo foi cumprido, pois os procedimentos descritos levaram o estudante a compreensão satisfatória das técnicas, características e considerações acerca de modelos de PRVs, como demonstrado pela descrição de três modelos diferentes na \cref{sec:PRVJT}, além da citação de outros modelos analisados. Focou-se, principalmente, na problemática de ônibus, introduzida na \cref{sec:urbanização e transporte público}, como motivação para a realização deste estudo.

O estudo computacional do problema tornou possíveis as implementações de modelos de roteirização. Com isso, obteve-se êxito na resolução de diferentes formulações do PCV. Já para PRVs, este relatório se limitou à modelagem do PRVJT de \textcite{VIEIRA:13}, pois a resolução de tais problemas renderia uma pesquisa própria, como indicam os grandes esforços dos artigos analisados para definir e implementar heurísticas sofisticadas. Ainda assim, foi possível aprender o básico sobre estes métodos, o que foi crucial para a compreensão e comparação dos \emph{solvers} testados.

Pode-se dizer que os principais objetivos desta pesquisa foram concluídos, criando o arcabouço teórico necessário para estudos de caso acerca de PRVs específicos e suas técnicas de resolução.